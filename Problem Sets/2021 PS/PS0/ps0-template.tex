% Template to use to complete Problem Set 0.
% Note that using this template is *optional*: it provides a nice foundation
% for getting started with LaTeX, but you aren't required to use it!
% If you are using Overleaf, you'll want to upload this file to your account.
% Before modifying this file, we recommend trying to compile it as-is
% to see what the basic template gives.

\documentclass[12pt]{article}

\usepackage{amsmath}
\usepackage[margin=2.5cm]{geometry}
% If you want to use this package, make sure to download it from the course
% website!
\usepackage{csc}

% Document metadata
\title{Insert title here}
\author{Insert author here}
\date{Insert date here}


% Document starts here
\begin{document}
\maketitle

\section*{My Courses}
Replace this text with a list of the courses you're taking.


\section*{Set notation}

% Fill in the following:

\[
    S_1 \cap S_2 = \dots
\]


\section*{A truth table}

Look at \code{sample\_latex.tex} for an example of a table.


\section*{A calculation}

Look at \code{sample\_latex.tex} for an example of \code{align*}.

\begin{align*}
    \log_x (3 \sqrt x) &= k \\
    % TODO: fill in steps here!
\end{align*}


\end{document}
