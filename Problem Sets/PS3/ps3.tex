\documentclass[12pt]{article}




\usepackage{amsmath}
\usepackage{amsthm}
\usepackage[margin=2cm,headheight=3ex]{geometry}
\usepackage{csc}
\usepackage{fancyhdr}
\usepackage{xcolor}
\usepackage{csquotes}
\usepackage[normalem]{ulem}
\usepackage{enumerate}
\renewcommand\labelenumi{(\alph{enumi})}

% \newcommand{\R}{\mathbb{R}}
% \newcommand{\N}{\mathbb{N}}
% \newcommand{\Z}{\mathbb{Z}}
\newcommand{\T}{\True}
\newcommand{\F}{\False}
\theoremstyle{definition}
\newtheorem{case}{Case}
\newtheorem*{induction}{Induction}

\pagestyle{fancyplain}
\lhead{\fancyplain{}{Junaid Arshad, Sadeed Ahmed, \& Frederick Meneses}}
\rhead{\fancyplain{}{CSC165 - Problem Set 3}}

% Document Metadata
\title{CSC165 - Problem Set 3}
\author{Junaid Arshad, Sadeed Ahmed, \& Frederick Meneses}
\date{March 18, 2022}

\begin{document}
\maketitle
%~~~~~~~~~~~~~~~~~~~~~~~~~~~~~~~~~~~~~~~~~~~~~~~~~~~~~~~~~~~~~~~
%~~~~~~~~~~~~~~~~~~~~~~~~~~~SECTION 1~~~~~~~~~~~~~~~~~~~~~~~~~~~
%~~~~~~~~~~~~~~~~~~~~~~~~~~~~~~~~~~~~~~~~~~~~~~~~~~~~~~~~~~~~~~~
\section{Number Representation}

\begin{enumerate}

    \item Let $(b_{k-1}\dots b_0)_2$ be a binary representation of a natural number, and $a$ be a single bit. The following is a proof that $(b_{k-1}\dots b_0 a)_2 = 2(b_{k-1}\dots b_0)_2 + a$:
    
        \begin{proof}
            Want to prove: $(b_{k-1}\ldots b_0a)_2 = 2(b_{k-1}\ldots b_0)_2 + a$.
            Let $(b_{k-1}\ldots b_0)_2 = \sum_{i=0}^{k-1}b_i2^i$.
            We will show the RHS = LHS.
            Let $k' = k + 1$ and $b_0' = a$ and $\forall i \in \{1,2,\ldots,k\}$, $b_i' = b_{i-1}$
            
            \begin{align*}
                2(b_{k-1}\ldots b_0)_2 + a &= \sum_{i=0}^{k-1}b_i2^{i+1} + a \\
                &= \sum_{i=1}^{k}b_{i-1}2^{i} + a \\
                &= \sum_{i=1}^{k'-1}b_{i}'2^{i} + a2^0 \tag{$2^0 = 1$}\\
                &= \sum_{i=1}^{k'-1}b_{i}'2^{i} + b_0'2^0 \tag{$b_0' = a$}\\
                &= \sum_{i=0}^{k'-1}b_{i}'2^{i} \\
                &= (b_{k'-1}'...b_0')_2 \\ 
                &= (b_{k}'....b_{1}'b_{0}')_2 = (b_{k-1}\ldots b_0a)_2
            \end{align*}
            
        \end{proof}
    
    \item Let $n \in \N$ and let $(b_{2n-1}\dots b_0)_2$ be the binary representation of a natural number. The following is a proof by induction that $(b_{2n-1}\dots b_0)_2 - (b_1 b_0)_2 - (b_3 b_2)_2 - \dots - (b_{2n-1} b_{2n-2})_2$ is a multiple of 3:
        
        \begin{proof}
            Let $n \in \N$ and $(b_{2n-1} b_{2n-2} \ldots b_0)_2 = \sum_{i=0}^{2n-1}b_i2^i$ be the binary representation of $n$.
            Let $$P(n): \exists k \in \Z, (b_{2n-1} b_{2n-2} \ldots b_0)_2 - (b_1b_0)_2 - (b_3b_2)_2 - \ldots - (b_{2n-1}b_{2n-2})_2 = 3k$$ \\
            \textbf{Base Case. }(n = 0)
            In this case, the binary representation of $n$ is $$\sum_{i=0}^{-1}b_i2^i = 0 = ()_2$$
            
            We want to show, $P(0): \exists k_0 \in \Z, ()_2 - ()_2 - \ldots - ()_2 = 3k_0$
            
            Let $k_0 = 0$
            $$()_2 - ()_2 - \ldots - ()_2 = 0 = 3(0) = 3k_0$$
            
            \textbf{Inductive Step. } WTS: $\forall m \in \N, P(m) \implies P(m+1)$ \\
            \textbf{Inductive Hypothesis: } Assume $$P(m): \exists d \in \Z, (b_{2m-1} b_{2m-2} \ldots b_0)_2 - (b_1b_0)_2 - (b_3b_2)_2 - \ldots - (b_{2m-1}b_{2m-2})_2 = 3d$$ 
            Explicitly, we want to show:
            $$P(m+1): \exists d_2 \in \Z, (b_{2m+1} b_{2m} \ldots b_0)_2 - (b_1b_0)_2 - (b_3b_2)_2 - \ldots - (b_{2m-1}b_{2m-2})_2 - (b_{2m+1}b_{2m})_2 = 3d_2$$
            We also know from the given hint that $$\forall m \in \N, \exists y \in \Z, 4^m - 1 = 3y $$
            Let $d_2 = 2y \cdot b_{2m+1} + y \cdot b_{2m} + d$\\
            We will show that the LHS = RHS.
            \begin{align*}
                &= (b_{2m+1} b_{2m} \ldots b_0)_2 - (b_1b_0)_2 - (b_3b_2)_2 - \ldots - (b_{2m-1}b_{2m-2})_2 - (b_{2m+1}b_{2m})_2 \\
                &= \sum_{i=0}^{2m+1} b_i2^i - (b_1b_0)_2 - (b_3b_2)_2 - \ldots - (b_{2m-1}b_{2m-2})_2 - (b_{2m+1}b_{2m})_2 \\
                &= \sum_{i=0}^{2m-1} b_i2^i + \sum_{i=2m}^{2m+1} b_i2^i  - (b_1b_0)_2 - (b_3b_2)_2 - \ldots - (b_{2m-1}b_{2m-2})_2 - (b_{2m+1}b_{2m})_2 \\
                &= \bigg(\sum_{i=0}^{2m-1} b_i2^i - (b_1b_0)_2 - \ldots - (b_{2m-1}b_{2m-2})_2 \bigg) + \bigg(\sum_{i=2m}^{2m+1} b_i2^i   - (b_{2m+1}b_{2m})_2\bigg) \\
                &= \big((b_{2m-1}b_{2m-2}\ldots b_0)_2 - (b_1b_0)_2 - \ldots - (b_{2m-1}b_{2m-2})_2\big) + \big(b_{2m+1} 2^{2m + 1} + b_{2m} 2^{2m} -  b_{2m+1} 2^{1} - b_{2m} 2^{0}\big) \\
                &= 3d + \big(b_{2m+1}2^{2m+1} - b_{2m+1}2\big) + \big(b_{2m}2^{2m} - b_{2m}2^{0}\big) \tag{by Inductive Hypothesis} \\ 
                % 2^{2m}(2^1b_{2m+1} + 2^0b_{2m}) + 3d - (b_{2m+1}b_{2m})_2 \tag{by Inductive Hypothesis}\\
                &= 3d + 2b_{2m+1}(4^{m} - 1) + b_{2m} (4^m - 1)  \\
                &= 3d + 2b_{2m+1}(3y) + b_{2m} (3y) \tag{using the given hint} \\
                &= 3(d + 2y \cdot b_{2m+1} + y \cdot b_{2m}) \tag{Factoring out 3}\\
                &= 3d_2
            \end{align*}
        \end{proof}
    
    \newcommand{\sm}[1]{\sum_{i=0}^{#1n-1}}
    \newcommand{\fb}[1]{(b_{#1} \dots b_0)_2}
    \newcommand{\mb}[2]{(b_{#1} b_{#2})_2}
    \newcommand{\mbb}[4]{(b_#1 \cdot 2^#2 + b_#3 \cdot 2^#4)}
    \item Let $x$ be a natural number with a binary representation where the difference between the number of 1 bits with an even index and the number of 1 bits with an odd index is a multiple of 3. The following is a proof that $x$ is a multiple of 3:
        
        \begin{proof}
            Let $x \in \N$. The binary representation of $x$ can have an even \emph{or} odd number of bits.
            If $x$ has an even number of bits, then its binary representation is of the form $(b_{2n-1} \dots b_0)_2$.
            If $x$ has an odd number of bits, then we can add a leading zero, in which case its binary representation is also $(0 b_{2n-2} \dots b_0)_2$, which can be similarly be re-written as $(b_{2n-1} \dots b_0)_2$.
            
            Given that the binary representation of $x$ is $(b_{2n-1} \dots b_0)$, assume that the difference between the number of 1 bits with an even index and the number of 1 bits with an odd index is a multiple of 3, i.e.:
            \begin{align*}
                3 \DIV \bigg(\sm{}b_{2i} - \sm{}b_{2i+1} \bigg),
            \end{align*}
            or that:
            \begin{align*}
                \exists d_1 \in \Z,\ \sm{}b_{2i} - \sm{}b_{2i+1} = 3d_1.
            \end{align*}
            
            From \textbf{Part B}, we know that:
            \begin{align*}
                \exists d_3 \in \Z, \fb{2n-1} - \mb{1}{0} - \mb{3}{2} - \dots \mb{2n-1}{2n-2} = 3d_3\\
            \end{align*}
            
            \emph{Want to prove:} $\fb{2n-1}$ is divisible by 3, or that, $3 \DIV \sm{2} b_i2^i$, i.e.: 
            \begin{gather*}
                \exists d_2 \in \Z, \sm{2}b_i2^i=3d_2
            \end{gather*}
            
            Let $d_2 = d_1 + d_3 + \sm{} b_{2i+1}$.
            
            \vspace{0.25cm}
            
            Using our result from \textbf{Part B}:
            \begin{gather*}
                \fb{2n-1} - \mb{1}{0} - \mb{3}{2} - \dots \mb{2n-1}{2n-2} = 3d_3\\
                \fb{2n-1} - \mbb{1}{1}{0}{0} - \mbb{3}{1}{2}{0} - \dots - (b_{2n-1} \cdot 2^1 + b_{2n-2} \cdot 2^0) = 3d_3\\
                \fb{2n-1} - (b_1 \cdot 2 + b_3 \cdot 2 + \dots + b_{2n-1} \cdot 2) - (b_0 \cdot 1 + b_2 \cdot 1 + \dots + b_{2n-2} \cdot 1) = 3d_3\\
                \fb{2n-1} - 2(b_1 + b_3 + \dots + b_{2n-1}) - (b_0 + b_2 + \dots b_{2n-2}) = 3d_3\\
                \fb{2n-1} + 2\sm{} b_{2i + 1} + \sm{} b_{2i} = 3d_3\\
                \sm{2} b_i 2^i = 3d_3 + 2 \sm{} b_{2i+1} + \sm{} b_{2i}
            \end{gather*}
            
            From our assumption, we know that:
            \begin{gather*}
                \sm{} b_{2i} - \sm{} b_{2i+1} = 3d_1 \iff \sm{} b_{2i} = 3d_1 + \sm{} b_{2i+1}
            \end{gather*}
            Plugging this value into our expression above:
            \begin{align*}
                \sm{2} b_i 2^i &= 3d_3 + 2 \sm{} b_{2i+1} + 3d_1 + \sm{} b_{2i+1}\\
                &= 3d_3 + 3d_1 + 3 \sm{} b_{2i+1}\\
                &= 3\bigg(d_3 + d_1 + \sm{} b_{2i+1}\bigg)\\
                &= 3d_2
            \end{align*}
            \emph{Thus, we have proven that $\fb{2n-1}$ is divisible by 3, or that, $3 \DIV \sm{2} b_i2^i$ when the difference between the number of 1 bits with an even index and the number of 1 bits with an odd index is a multiple of 3.}
        \end{proof}
    
\end{enumerate}
%~~~~~~~~~~~~~~~~~~~~~~~~~~~~~~~~~~~~~~~~~~~~~~~~~~~~~~~~~~~~~~~
%~~~~~~~~~~~~~~~~~~~~~~~~~~~SECTION 2~~~~~~~~~~~~~~~~~~~~~~~~~~~
%~~~~~~~~~~~~~~~~~~~~~~~~~~~~~~~~~~~~~~~~~~~~~~~~~~~~~~~~~~~~~~~
\newpage
\section{Induction}
\setcounter{equation}{0}
For $m, n \in \Z^+$, define $P(m, n)$ to be:
\begin{align*}
    \text{the number of ways to write $n=x_1 + \dots + x_m$ with $x_1, \dots x_m \in \N$ is $\frac{(n+m-1)!}{n!(m-1)!}$}.
\end{align*}

\begin{enumerate}
    \item We will prove the following statements:
    
    \begin{enumerate}[i]
    
        \item $\forall n \in \Z^+, P(1, n)$
            \begin{proof}
                Let $n \in \Z^+$. \textbf{WTS:} $P(1, n)$: The number of ways to write $n = x_{1} + \ldots + x_{m}$ with $x_1, \dots x_m \in \N$ when $ m = 1$ is 1 \\ 
                Let $m = 1, n = n$. In this case, $x_m = x_1 = n$. Thus, we can see that the number of ways to write n is 1. We can also confirm this using the formula:
                $$\frac{(n+1-1)!}{n!(1-1)!} = \frac{n!}{n!} = 1$$
            \end{proof}
        
        \item $\forall m \in \Z^+, P(m, 1)$
            \begin{proof}
                Let $m \in \Z^+$. \textbf{WTS:} $P(m, 1)$: The number of ways to write $1 = x_1 + \cdots + x_m$ with $x_1, \ldots, x_m \in \N$ is m \\
                Since $n = 1$ here, one of $x_1, \ldots, x_m \in \N$ must be 1 whereas every other digit must $ = 0$, thus there are $m$ ways to represent 1 this way since any one of the $m$ digits can take on the value 1 and all others must be $=0$. Conversely, if one of the $m$ digits is 1 and all the others are $0$, they must add up to $n = 1$. We can also show this using the given formula, \\
                %Let $k \in \Z^+$ and assume that $1 \leq k \leq m$ \\
                %Let $x_k = 1$ and let all other $x_i = 0$ where $i \in \Z^+$ and $1 \leq i \leq m$ and $i \neq k$ \\ 
                %Since $k$ can take on any value between 1 and m inclusive, the number of ways to write $1$ with $x_1, \ldots, x_m \in \N$ is $m$. We can also very this using the given formula,
                $$\frac{(1+m-1)!}{1!(m-1)!} = \frac{m!}{(m-1)!} = m$$
                %'$1$' can be in any index once while 0's are in every other index.
            \end{proof}
            
        \item $\forall m, n \in \Z^+, P(m+1, n) \land P(m, n+1) \implies P(m+1, n+1)$
            \begin{proof}
            Assume $P(m + 1,n)$: The number of ways to write $n = x_1 + x_2 + \cdots + x_{m+1}$ with $x_1, \dots x_{m+1} \in \N$ using $m+1$ terms is $$\frac{(n+m)!}{n!m!}$$ and assume $P(m, n+1)$: The number of ways to write $n + 1= x_1 + x_2 + \cdots + x_{m}$ with $x_1, \dots x_m \in \N$ with $m$ terms is $$\frac{(n + m)!}{(n+1)!(m-1)!}$$ \\
            We want to show:$$P(m+1, n+1): \text{the number of ways to write } n + 1 = x_1 + x_2 + \cdots + x_{m+1}$$ $$ \text{ for some } x_1, x_2, \ldots, x_{m+1} \in \N \text{ is } = \frac{(n+1+m)!}{(n+1)!m!}$$ \\ 
            Since $x_1$ is a natural number, we know $x_1 \geq 0 \leftrightarrow x_1 = 0 \lor x_1 \geq 1$. Using this we can split up our domain of $x_1$ into two independent parts. \\ 
            \\ 
            %This proof relies on our choices for $x_1 \in \N$, so we will break this proof into cases. 
            \textbf{\underline{When $x_1 = 0$:}} \\ 
            Starting with the LHS of \textbf{WTS:} \\ 
            $n + 1 = x_1 + x_2 + \cdots + x_{m+1}$ \\
            Define $x_i'$ such that $x_i' = x_{i+1}$ for $1 \leq i \leq m$.
            $$n + 1 = 0 + x_2 + \cdots + x_{m+1} = x'_1 + x'_2 + \cdots + x'_m$$
            Since there are $m$ terms in this equation, by our assumption that $P(m, n + 1)$ is True the number of ways to write $n + 1$ with $m$ terms is $$\frac{(n + m)!}{(n+1)!(m-1)!}$$
            \textbf{\underline{When $x_1 \geq 1$:}} \\
            In this situation, we consider what we want to prove:
            \begin{align*}
                n + 1 &= x_1 + x_2 + \cdots + x_{m+1}\\
                n &= x_1 - 1 + x_2 + \cdots + x_{m+1} \tag{subtracting 1 from both sides} \\
                n &= (x_1 - 1) + x_2 + \cdots + x_{m+1} 
            \end{align*}
            We know: $x_1 \geq 1 \leftrightarrow x_1 - 1 \geq 0 \text{ } (\in \N)$  \\
            We then define $x_1'' = x_1 - 1$ and $x_i''$ such that $x_i'' = x_{i}$ for $2 \leq i \leq m+1$. Then, we have the equation
            $$n = x_1' + x_2' + \cdots + x_{m+1}'$$
            Since there are $m+1$ terms in this equation, by our assumption that $P(m+1, n)$ is True, the number of ways to write $n$ using $m+1$ terms is: 
            $$\frac{(n+m)!}{n!m!}$$
            
            Therefore, the total number of solutions we have for the equation $n+1 = x_1 + x_2 + \cdots + x_{m+1}$ when $x_1 \geq 0$ will be the sum of the solutions to the same equation when $x_1 =0$ and when $x_1 \geq 1$
            \begin{align*}
                P(m, n + 1) + P(m + 1, n)
                &= \frac{(n + m)!}{(n+1)!(m-1)!} + \frac{(n+m)!}{n!m!}\\
                &= \frac{(n+1)(n + m)! + m(n + m)!}{(n+1)n!m(m-1)!} \\
                &= \frac{(n+1+m)(n + m)!}{(n+1)!m!} \\
                &= \frac{(n+1+m)!}{(n+1)!m!} \\
                &= P(m + 1, n + 1)
            \end{align*}
            Which is exactly what we wanted to show.
            \end{proof}
        
    \end{enumerate}
    
    \item We will use the results from part (a) to prove $P(2,2) \land P(3, 3)$.
        \begin{proof}
            From part $a$ we know that $\forall n \in Z^+, P(1, n)$ and  $\forall n \in \Z^+, P(m, 1)$ are True. Using these two facts, we know that the following are true: 
            \begin{itemize}
                \item $P(1, 2)$
                \item $P(1, 3)$
                \item $P(2, 1)$
            \end{itemize}
            %Thus we take n = 2 to conclude $P(1, 2)$ is \texttt{True}.\\
            % From ii) we know that $\forall n \in \Z^+, P(m, 1)$. Thus we take m = 2 to conclude $P(2, 1)$ is \texttt{True}.\\
            From iii) we know that $\forall m, n \in \Z^+, P(m+1, n) \land P(m, n+1) \implies P(m+1, n+1)$. Using this and the true statements from above we can conclude the following: \\ 
            \begin{align*}
                P(1, 2) \land P(2, 1) \implies P(2, 2) \\
                P(3, 1) \land P(2, 2) \implies P(3, 2) \\ 
                P(2, 2) \land P(1, 3) \implies P(2, 3) \\ 
                P(3, 2) \land P(2, 3) \implies P(3, 3) \\
            \end{align*}
            %Since $P(1, 2)$ and $P(2, 1)$ are both \texttt{True}, we can conclude that so is $P(2, 2)$.
            %\\  
            %Without loss of generality, 
            %we may conclude $P(1, 3)$ and $P(3, 1)$ are \texttt{True}. Hence, $P(3, 3)$ is True.
            Thus, $P(3, 3)$ is also True. This completes the proof.
        \end{proof}
    
    \item For $t \in \Z^+$ with $t \geq 2$, define $Q(t)$ to be: $\forall m, n \in \Z^+, m + n = t \implies P(m, n)$. We will prove the following by induction: $\forall t \in \Z^+, t \geq 2 \implies Q(t)$.
        \begin{proof} 
            \textbf{Base Case. }(t = 2)  \\ 
            The only possible values that $m$ and $n$ can have in this case are $m = 1$ and $n = 1$. We know that $P(1, 1)$ is $\True$ using our results from \textbf{(1.a.i)} \\
            \\ 
            \textbf{Inductive Step. }Let $t_1 \in \N$. Assume $t_1 \geq 2$ and also assume $Q(t_1)$: $\forall m_1, n_1 \in \Z^+, m_1 + n_1 = t_1 \implies P(m_1, n_1)$ \\
            \textit{WTS:} $Q(t_1 + 1)$: $\forall m_2, n_2 \in \Z^+, m_2 + n_2 = t_1 + 1 \implies P(m_2, n_2)$ 
            \begin{itemize}
                \item Let $m_2, n_2 \in \Z^+$
                \item Assume $m_2 + n_2 = t_1 \leftrightarrow m_2 +n_2 -1  = t_1$. \item Since by our assumption, $t_1 \geq 2$, the values that $m_2$ and $n_2$ can take on can be split up into 3 cases 
                \case{$m_2 = 1 (\in \Z^+)$} \\ 
                Since this is the smallest value that $m_2$ can have, in this case for the expression $m_2 + n_2 - 1$ to be equal to $t_1$, it must stand that $n_2\geq 2$.
                
                Since $m_2 = 1$ and $n_2 \in \Z^+$, we know from \textbf{(2.a.i)} that $P(m_2, n_2) = P(1, n_2)$ is \True.
                
                \case{$n_2 = 1 (\in \Z^+)$} \\ 
                Since this is the smallest value that $n_2$ can have, in this case for the expression $m_2 + n_2 - 1$ to be equal to $t_1$, it must stand that $m_2 \geq 2$.
                Thus, we have that $n_2 = 1$ and $m_2 \geq 2$ in this case.
                
                Since $n_2 = 1$ and $m_2 \in \Z^+$, we know from \textbf{(2.a.ii)} that $P(m_2, n_2) = P(m_2, 1)$ is \True.
                
                \case{$n_2 > 1 \land m_2 > 1$}
                In this case, the following statements are $\True$ using the induction hypothesis:
                \begin{align*}
                    m_2 + (n_2 - 1) = t_1 \implies P(m_1, n_1 - 1) \tag{$m_2, (n_2-1) \in \Z^+$}\\
                    (m_2 - 1) + n_2 = t_1 \implies P(m_2, n_2 - 1) \tag{$(m_2-1), n_2 \in \Z^+$} \\
                \end{align*}
                Since both of these expressions are $\True$ in this case, by \textbf{(2.a.iii)}, we can conclude that:
                \begin{align*}
                    P(m_2 - 1, n_2) \land P(m_2, n_2 - 1) \implies P(m_2, n_2)
                \end{align*}
            \end{itemize}
            %\emph{Thus, we have proven that $t \geq 2$ is $\True$, then $Q(t)$ is also $\True$.}
        \end{proof}
    \item We will use the results from previous parts to prove $\forall m, n \in \Z^+, P(m, n)$.
        \begin{proof}
            Let $m, n \in \Z^+$.
            %Choose $t = m + n$. Since $m \geq 1 \land n \geq 1$, $t \geq 2$. Hence by 2c, we conclude Q(t). That is, $\forall m, n \in \Z^+, m + b = t \implies P(m, n)$, which is what we wanted to show.
            We know that $m + n \geq 2$, and from \textbf{(2.c)}, we know that this implies $Q(m+n)$, i.e.:
            \begin{align*}
                \forall m', n' \in \Z^+, m' + n' = m + n \implies P(m', n')
            \end{align*} 
            Consider the instance when $m' = n \land n' = n$. In this case, $P(m', n') = P(m, n)$, which is what we wanted to prove.
        \end{proof}
\end{enumerate}

\newpage
%~~~~~~~~~~~~~~~~~~~~~~~~~~~~~~~~~~~~~~~~~~~~~~~~~~~~~~~~~~~~~~~
%~~~~~~~~~~~~~~~~~~~~~~~~~~~SECTION 3~~~~~~~~~~~~~~~~~~~~~~~~~~~
%~~~~~~~~~~~~~~~~~~~~~~~~~~~~~~~~~~~~~~~~~~~~~~~~~~~~~~~~~~~~~~~
\section{Asymptotic Notation}
%~~~~~~~~~~~~~~~~~~~~~~VARIABLE DEFINITION~~~~~~~~~~~~~~~~~~~~~~%
\newcommand{\nn}{\ceil{\frac{1 + \sqrt{1 + 4k}}{2}}}
\newcommand{\nnf}{\frac{1 + \sqrt{1 + 4k}}{2}}
\newcommand{\nnfs}{\frac{1 - \sqrt{1 + 4k}}{2}}
\newcommand{\nnsq}{\sqrt{1 + 4k}}
\newcommand{\nnsqs}{\sqrt{1 - 4k}}

The following facts may be referenced in this section:
\begin{align*}
    &\forall n \in \Z, n \leq 2^n \tag{\textbf{Fact 1}}\\
    &\forall x, y \in \R^+, x \leq y \iff log_2(x) \leq log_2(y) \tag{\textbf{Fact 2}}\\
    &\forall x, y \in \R, x \leq y \iff 2^x \leq 2^y \tag{\textbf{Fact 3}}
\end{align*}

\begin{enumerate}

    \item We will \textbf{prove} that $log_2(k+n) \in \bigO(log_2 n)$, or that:
    \begin{align*}
        \exists c, n_0 \in \R^+, \forall n \in \N, n \geq n_0 \implies \log_2(k+n) \leq c \cdot \log_2(n)
    \end{align*}
        \begin{proof}
            Let $k$ be an arbitrary $\R^+$. 
            Let $c = 2$, where $c \in \R^+$. 
            Let $n_0 = \nn$, where $n_0 \in \R^+$.
            
            Assume that $n \geq n_0 \iff n \geq \nn$
            
            \emph{Want to prove:} $\log_2(k+n) \leq \log_2(n^2)$.
            
            Since $k > 0$,
            \begin{align*}
                k&>0\\[1ex]
                1 + 4k &> 1\\[1ex]
                \nnsq &> 1\\[1ex]
                1 - \nnsq &< 0\\[1ex]
                \nnfs &< 0\tag{\textbf{4}}
            \end{align*}
            We also know that:
            \begin{align*}
                \nnf > \nnfs
            \end{align*}
            
            Using our assumption and \textbf{(4)}, we can conclude following:
            \begin{align*}
                n - \bigg(\nnf\bigg) &> 0\\[1ex]
                n - \bigg(\nnfs\bigg) &> 0
            \end{align*}
            
            Moreover, since the aforementioned terms are greater than 0, we can perform the following arithmetic:
            \begin{align*}
                \bigg(n - \bigg(\nnf\bigg) \bigg) \cdot \bigg(n - \bigg(\nnfs\bigg) \bigg) &\geq 0\\[1ex]
                n^2 - n \cdot \bigg(\nnfs\bigg) - n \cdot \bigg(\nnf\bigg) + \bigg(\nnf\bigg) \cdot \bigg(\nnfs\bigg) &\geq 0\\[1ex]
                n^2 + \frac{-n + n \cdot \nnsq - n - n \cdot \nnsq}{2} + \frac{1^2 - (\nnsq)^2}{4} &\geq 0\\[1ex]
                n^2 - \frac{2n}{2} + \frac{1 - 1 - 4k}{4} &\geq 0\\[1ex]
                n^2 - n - k &\geq 0\\[1ex]
                n^2 - n - k &\geq 0\\[1ex]
                n^2 &\geq n + k
            \end{align*}
            Using \textbf{Fact (2)}, we can also conclude that:
            \begin{align*}
                %n^2 &\geq n + k \iff \log_2(n^2) \geq \log_2(n+k)\\
                n^2 \geq n + k &\iff \log_2(n+k) \leq \log_2(n^2) \\
                &\iff \log_2(n+k) \leq 2 \cdot \log_2(n) \\
                &\iff \log_2(n+k) \leq c \cdot \log_2(n) \\
            \end{align*}
            \emph{Thus, we have proven that:  $\log_2(k+n) \leq \log_2(n^2)$.}
        \end{proof}
    %~~~~~~~~~~~~~~~~~~~~~~VARIABLE DEFINITION~~~~~~~~~~~~~~~~~~~~~~%
    \newcommand{\e}{\epsilon}
    \newcommand{\bbn}{\ceil{\frac{1}{c^{\frac{1}{\e}}}+ n_0 + 1}}
    \newcommand{\bb}{\frac{1}{c^{\frac{1}{\e}}}}
     
    \item We will \textbf{disprove} that $n \in \Omega(n^{1+\e})$, or that: 
        \begin{align*}
            \forall c, n_0 \in \R^+, \exists n \in \N, n \geq n_0 \land n < c \cdot n^{1 + \e}
        \end{align*}
        \begin{proof}
            Let $\e, c, n_0 \in \R^+$.
            Let $n = \bbn \in \N$.
            
            \emph{Want to prove: }$n \geq n_0$ and that $c < c \cdot n^{1 + \e}$.
            
            \proofheader{Part 1:} We will prove that $n \geq n_0$.
            
            Since our choice of $n$ depends on $\bb$, we will show that $\bb > 0$:
            \begin{align*}
                \e &> 0\\
                \frac{1}{\e} &> 0\\
                c^{\frac{1}{\e}} &> 1\\
                0 < \bb &< 1
            \end{align*}
            \emph{Thus, since $n$ is the ceiling of the sum of positive real numbers (including $n_0$), we can conclude that $n \geq n_0$.}
            
            \newpage
            
            \proofheader{Part 2:} We will prove that $n < c \cdot n^{1 + \e}$.
            
            % PROVE THAT c/ne IS POSITIVE REAL
            Before continuing with the proof, we will prove that $\frac{c}{n^\e}$ is a positive real number. Since $\e$ is a positive real number, we can perform the following arithmetic operations:
            \begin{align*}
                \e &> 0\\
                n^\e &> 1\\
                0 < \frac{1}{n^\e} &< 1 \tag{\textbf{6}}
            \end{align*}
            Moreover, since $c$ is a positive real number and \textbf{(6)} is True, we can conclude that $\frac{c}{n^\e} > 0$.
            
            By our choice of $n$, and from our conclusion in \textbf{part 1} that $n$ consists of positive terms \big(including $\bb$\big), we can perform the following arithmetic operations:
            \begin{align*}
                n &> \bb\\[1ex]
                n^\e &> \frac{1}{c} \tag{raise inequality to the power of $\e$}\\
                c &> \frac{1}{n^\e} \tag{multiply by $\frac{c}{n^\e} > 0$}\\
                c &> n^{-\e} \\
                c &> n^{1 - 1 - \e}\\
                c &> n^{1 - (1 + \e)}\\
                c &> \frac{n}{n^{1 + \e}}\\
                c \cdot n^{1 + \e} &> n \tag{since $n^{1 + \e} > 0$}\\
            \end{align*}
            
            \emph{Thus, we have proven that $n \geq n_0$ and that $n < c \cdot n^{1 + \e}$.}
        \end{proof}
        
\end{enumerate}

\newpage
%~~~~~~~~~~~~~~~~~~~~~~~~~~~~~~~~~~~~~~~~~~~~~~~~~~~~~~~~~~~~~~~
%~~~~~~~~~~~~~~~~~~~~~~~~~~~SECTION 4~~~~~~~~~~~~~~~~~~~~~~~~~~~
%~~~~~~~~~~~~~~~~~~~~~~~~~~~~~~~~~~~~~~~~~~~~~~~~~~~~~~~~~~~~~~~
\section{More Asymptotic Notations}

    \begin{enumerate}
        \item We will \textbf{prove} that if $f + g \in \mathcal{O}(h)$, then $f \in \mathcal{O}(h)$ and $g \in \mathcal{O}(h)$.
            
            \begin{proof}
                Let $f, g, h : \N \xrightarrow{} \R^{\geq 0}$.
                Assume that $f + g \in \bigO (h)$, or that:
                \begin{align*}
                    \exists c_0, n_0 \in \R^+, \forall n \in \N, n \geq n_0 \implies (f+g)(n) \leq c \cdot h(n)
                \end{align*}
                
                We know that the definition of the sum of $f$ and $g$ is:
                \begin{align*}
                    \forall n \in \N, (f + g)(n) = f(n) + g(n)
                \end{align*}
                
                \emph{Want to prove:} 
                
                \textbf{$f \in \bigO(h)$:} $\exists c_1, n_1 \in \R^+, \forall n \in \N, n \geq n_1 \implies f(n) \leq c_1 \cdot h(n)$
                
                \textit{and}
                
                \textbf{$g \in \bigO(h)$:} $\exists c_2, n_2 \in \R^+, \forall n \in \N, n \geq n_2 \implies g(n) \leq c_2 \cdot h(n)$
                
                \vspace{0.25cm}
                \proofheader{Proof that $f \in \bigO(h)$}, i.e., $\exists c_1, n_1 \in \R^+, \forall n \in \N, f(n) \leq c_1 \cdot h(n)$.
                
                From our assumption, we know that $f(n) + g(n) \leq c_0 \cdot h(n)$.
                
                Let $c_1 = c_0$.
                Let $n_1 = n_0$.
                Let $n \in \N$.
                
                Since $n \geq n_1 = n_0$, using our assumption we know that $f(n) + g(n) \leq c_0 \cdot h(n)$.
                
                Since $g(n) \in \R^{\geq 0}$ (by the definition of the function), we know that if we subtract $g(n)$ from the left side, it would become even smaller and the inequality would still hold, i.e.:
                \begin{align*}
                    f(n) + g(n) \leq& c_0 \cdot h(n)\\
                    f(n) \leq& c_0 \cdot h(n) \tag{subtract $g(n)$ from the left side}\\
                    f(n) \leq& c_1 \cdot h(n) \tag{since $c_1 = c_0$}
                \end{align*}
                
                \vspace{0.25cm}
                \proofheader{Proof that $g \in \bigO(h)$}, i.e., $\exists c_2, n_2 \in \R^+, \forall n \in \N, g(n) \leq c_2 \cdot h(n)$.
                
                From our assumption, we know that $f(n) + g(n) \leq c_0 \cdot h(n)$.
                
                Let $c_2 = c_0$.
                Let $n_2 = n_0$.
                Let $n \in \N$.
                
                Since $n \geq n_2 = n_0$, using our assumption, we know that $f(n) + g(n) \leq c_0 \cdot h(n)$.
                Since $f(n) \in \R^{\geq 0}$ (by the definition of the function), we know that if we subtract $f(n)$ from the left side, it would become even smaller and the inequality would still hold, i.e.:
                \begin{align*}
                    f(n) + g(n) \leq& c_0 \cdot h(n)\\
                    g(n) \leq& c_0 \cdot h(n) \tag{subtract $f(n)$ from the left side}\\
                    g(n) \leq& c_1 \cdot h(n) \tag{since $c_1 = c_0$}
                \end{align*}
                
                \emph{Thus, we have proven that $f(n) \leq c_1 \cdot h(n)$ and $g(n) \leq c_2 \cdot h(n)$.}
            \end{proof}
        
        \newpage    
        \item For all functions $f, g: \N \xrightarrow{} \R^{\geq 0}$, we define the \textit{product function} $f \cdot g : \N \xrightarrow{} \R^{\geq 0}$ as follows:
            \begin{align*}
                \forall n \in \N, (f \cdot g)(n) = f(n) \cdot g(n).
            \end{align*}
        Let $f, g, h: \N \xrightarrow{} \R^{\geq 0}$. 
        We will \textbf{disprove} that if $f \cdot g \in \bigO(h)$, then $f \in \bigO(h)$ and $g \in \bigO(h)$.
        
        \vspace{0.25cm}
        \emph{Want to prove:} $\exists f, g, h : \N \xrightarrow{} \R^{\geq 0}, f \cdot g \in \bigO (h) \land \big(f \notin \bigO(h) \lor g \notin \bigO(h)\big)$
        
            \begin{proof}
                Let $n \in \N$.
                Let $g(n) = n^2$.
                Let $h(n) = n$.
                Let $f(n)$ be the piece-wise function defined by:
                \[   \left\{
                \begin{array}{ll}
                      f(n) = 0 & \text{when } n = 0 \\
                      f(n) = \frac{1}{n} & \text{when }n \geq 1 \\
                \end{array} 
                \right. \]
                
                \emph{Want to prove:} $f \cdot g \in \bigO (h) \land \big(f \notin \bigO(h) \lor g \notin \bigO(h)\big)$
                
                \vspace{0.25cm}
                \proofheader{Proof that $f \cdot g \in \bigO(h)$}, i.e., $\exists c_0, n_0 \in \R^+, \forall n \in \N, n \geq n_0 \implies (f \cdot g)(n) \leq c_0 \cdot h(n)$.
                
                By definition of product functions, we know $(f \cdot g)(n) = f(n) \cdot g(n)$.
                
                Let $c_0 = 1$.
                Let $n_0 = 1$.
                Let $n \in \N$.
                
                From our assumption that $n \geq n_0$, we can infer the following:
                \begin{flalign*}
                    n &\geq 1\\[1ex]
                    \iff 0 < \frac{1}{n} &\leq 1 \tag{\textbf{1}}\\[1ex]
                    \text{Moreover, from our assumption:}&&\\
                    n^2 &\geq 1 > 0 \tag{\textbf{2}}
                \end{flalign*}
                
                We can then conclude the following:
                \begin{align*}
                    f(n) \cdot g(n) = \frac{1}{n} \cdot n^2 \geq 0 \tag{using \textbf{(1)} and \textbf{(2)}}
                \end{align*}
                \emph{Want to prove:} $f(n) \cdot g(n) \leq c_0 \cdot h(n)$.
                
                We can perform arithmetic operations on the product functions to prove this:
                \begin{align*}
                    f(n) \cdot g(n) &= \frac{1}{n} \cdot n^2\\
                    &= 1\cdot n^{2 - 1}\\
                    &= n\\
                    &\leq 1 \cdot n\\
                    &=c_0 \cdot h(n) \tag{from our choice of $c_0$ and $h(n)$}
                \end{align*}
                
                \proofheader{Proof that $f \notin \bigO(h) \lor g \notin \bigO(h)$:}
                
                \emph{Want to prove:} $g \notin \bigO (h)$, i.e.,
                \begin{align*}
                    \forall c_1, n_1 \in \R^+, \exists n \in \N, n \geq n_1 \land g(n) \geq c_{1} \cdot h(n)
                \end{align*}
                Let $c_1, n_1 \in \R^+$.
                Let $n = \ceil{c_1 + n_1 + 1}$, where $n \in \N$.
                
                \emph{Want to prove:} $n \geq n_1 \land g(n) \geq c_1 \cdot h(n)$.
                
                By our choice of $n$, we know that $n \geq n_1$ is $True$.
                
                By our choice of $n$, we also know the following:
                \begin{align*}
                    c_1 &\leq n\\
                    c_1 \cdot n &\leq n^2\\
                    c_1 \cdot h(n) &\leq g(n)
                \end{align*}
                \emph{Thus, we have proven that $f \cdot g \in \bigO(h)$ and that $g \notin \bigO (h)$.}
            \end{proof}
    \end{enumerate}
\end{document}